\subsection{Authors' Biographies}

% add a few word about TRAIL, ARIAC & the workshop in Nantes

\subsubsection{Julien Albert}

After an initial career as a librarian, Julien Albert embarked on a career change and obtained a master's degree in computer science from UNamur in 2020. He then worked for one year at UNamur on the EFFaTA-MeM research project, which aims to develop innovative tools for text analysis. In September 2021, he began a Ph.D. in computer science at UNamur under the supervision of Professors Benoît Frenay and Bruno Dumas. His research area is explainability in artificial intelligence. His approach involves placing the user at the centre of concerns by combining explainability techniques from machine learning with methods developed in human-computer interaction.

\subsubsection{Martin Balfroid}

Martin Balfroid is a PhD student at the University of Namur, his research investigates AI-in-the-loop approaches to improve software engineering. He earned his master's in Computer Science, focusing on Data Science, in June 2022. The results of his master's thesis were published at the 2nd Software Testing Education Workshop. Martin began his PhD in July 2022 with funding from the ARIAC project and is supervised by Assistant Professors Benoît Vanderose and Xavier Devroey. 

\subsubsection{Miriam Doh}

Miriam Doh obtained a master's degree in Information and Communication Engineering from the University of Trento (UniTn) in Italy in 2021. Her master's thesis focused on the application of genetic algorithms to social networks, with a focus on studying the problem of community segregation in metropolitan areas. After completing her degree, she began a joint PhD program between ULB and UMONS on the intersection of Deep Learning and Computer Vision, with a particular emphasis on Explainable AI (XAI). Her research project is dedicated to exploring the integration of Cognitive Psychology principles to advance Explainable and Trustworthy Artificial Intelligence, particularly within the context of Face Analysis applications.

\subsubsection{Jeremie Bogaert}

Jérémie Bogaert obtained his master's degree in computer science engineering with a focus on artificial intelligence from UCLouvain in 2021. His master's thesis explored the limitations of deep fake news generation models and their detection using machine learning models and human readers. He started his doctoral thesis at UCLouvain in September 2021 and is currently working on the interaction between interpretable machine learning models and human readers for the detection of deep fake news.

\subsubsection{Luca La Fisca}

Luca La Fisca is currently a PhD student with a keen interest in neural engineering. His primary research focus revolves around advancing tools for a deeper understanding of the intricacies of the human brain. Luca's doctoral thesis specifically delves into the realm of ElectroEncephalogram (EEG) analysis. He is particularly fascinated by the interpretation of latent space to unveil critical interactions among brain regions during the execution of specific tasks, with a primary emphasis on visual tasks. Additionally, Luca harbours a strong interest in the field of neurofeedback.

Within the ARIAC project, Luca La Fisca is actively involved in Work Package 1, which centres on the interactions between humans and artificial intelligence. His contributions span various aspects, including interactive and human-in-the-loop algorithms, user assistance in AI-in-the-loop scenarios, consensus mechanisms, handling imperfect multi-expert labels, and the development of explainable AI solutions.

\subsubsection{Liesbet De Vos}

Liesbet De Vos obtained a master's in Linguistics at the Catholic University of Leuven in 2021. Fascinated by computational linguistics, she completed her studies with an advanced master's in Artificial Intelligence at the Catholic University of Leuven, which she completed in 2022. Liesbet continues to nurture her passion for language during a PhD at the University of Namur, where she focuses on building hybrid AI systems that learn to use language through the same mechanisms as humans. In her thesis, she aims to extend the computational construction grammar framework to the visual modality so that it can adequately represent and learn the linguistic structure of sign languages. Within the ARIAC project, Liesbet actively contributes to Work Package 2, which revolves around trust mechanisms for artificial intelligence. 

\subsubsection{Bryan Renard}

Bryan Renard obtained a master's degree in theoretical physics from UNamur in 2022. He then changed his career path and is now a dedicated PhD student whose research interests span several exciting domains within the field of artificial intelligence. His primary focus is the application of artificial intelligence in the realm of proteins, exploring innovative ways to harness AI (especially LLMs) for protein-related research. Additionally, Bryan is passionate about self-supervised learning, particularly in the context of Automatic Speech Recognition (ASR).

His thesis is jointly conducted by UNamur and Multitel. It is funded by the FoodWal portfolio from the Public Service of Wallonia (Economy, Employment, and Research), more particularly within the PEPTIBoost project. As a part of the ARIAC project, Bryan Renard plays an integral role in Work Package 4, which revolves around optimizing AI implementations. His contributions encompass a wide range of topics, including transfer learning, High-Performance Computing (HPC) and self-supervised learning techniques.

\subsubsection{Vincent Stragier}

Vincent Stragier is a PhD student at the University of Mons (UMONS). He is working on an interactive assistant for visually impaired and blind people within the ISIA Lab, a department of the Faculty of Engineering. His research interests are mainly focused on NLP, large language models and computer vision related topics.

In 2021, he obtains his master’s degree in electrical Engineering, specialized in Signals, Systems and BioEngineering from the Faculty of Engineering in Mons. In 2020, he works on an epilepsy detection pipeline base on an XGBoost classifier built by the CETIC, where he is Engineer Intern at the time. During his studies, he participates in the electronic student association, electroLAB, and the Erasmus Student Network of Mons, ESNMons. In his free time, he likes taking photographs, fixing various things (hardware and software related), and learning new skills.

\subsubsection{Emmanuel Jean}

In 2009, Emmanuel Jean earned a dual degree in electrical engineering from the Faculty of Engineering at the University of Mons and Supelec-Paris. Subsequently, he joined the Signal Processing and Embedded Systems department at Multitel, where he actively participated in various regional and European projects involving vocal technologies and multimodal Human-Computer Interaction (HCI).

In 2012, he furthered his education by obtaining a Bachelor's degree in Management Sciences from the Louvain School of Management at UCL-Mons. Since 2017, his professional focus has shifted towards diverse projects centred around Deep Learning applied to temporal signals, including audio, speech, and vibrations. His current research interests revolve around the development of weakly supervised machine learning techniques and the deployment of reliable artificial intelligence systems.
